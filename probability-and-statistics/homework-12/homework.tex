\documentclass{article}
\usepackage{ctex}
\usepackage{float}
\usepackage{geometry}
\usepackage{amsmath}
\usepackage{graphicx}
\geometry{a4paper, scale=0.8}

\title{概率论与数理统计第十二次作业}
\author{ZhaohengLi 2017050025}

\begin{document}
\maketitle

\section{5.3.26}
总体的分布函数为:

$$F(x)=\int^x_06t(1-t)dt=3x^2-2x^3=x^2(3-2x),0\leq x \leq 1$$

$$1-F(x)=(1-x)^2(2x+1),0\leq x \leq 1$$

因此样本中位数$m_{0.5}=x_{(5)}的精确分布密度函数为$

$$p_{m_{0.5}}(x)=3780x^9(1-x)^9(3-2x)^4(2x+1)^4$$

通过近似计算可以得到:

$$p(x_{0.5})=6*0.5*(1-0.5)=1.5$$

当$n=9$时,$m_{0.5}$的渐进分布为:

$$m_{0.5}\sim N(x_{0.5},\frac1{4np^2(x_{0.5})})=N(0.5,\frac1{81})$$

最终得出:

$$P(m_{0.5}<0.7)\approx \Phi (1.8)=0.9641$$

\section{5.3.29}

(1)

$$E\{F(x_{(6)}\}=\frac6{11}$$
$$Var\{F(x_{(6)}\}=\frac{6(10+1-6)}{(10+1)^2(10+2)}=\frac{5}{242}$$

(2)

因为$F(x_{(6)})\sim Be(6,5)$,所以$F(x_{(6)}$在$x=0.15$处的分布函数值为:

$$betacdf(0.15,6,5)=0.0014$$

\section{5.3.31}
令 $y_i=\frac{x_i-\mu}{\sigma}\simExp(1)$,那么$y_{(1)},\cdots,y_{(n)}$的联合密度为:

$$p(y_1,\cdots,y_n)=n!exp\{-\sum^n_{i=1}y_i\}$$

通过变换求得联合密度为$f(t_1,\cdots,t_n)=exp\{-\sum^n_{i=1}t_i\}$,由联合密度可以知道$T_1,\cdots,T_n$是独立同分布的随机变量,且$T_i\sim Exp(1)$,因此有:

$$P((n-i-1)\frac2\sigma(x_{(i)}-x_{(i-1)})\leq x)=P(2(n-i-1)(y_{(i)}-y_{(i-1)})\leq x)$$

$$=P(2T_i\leq x)=P(T_i \leq \frac x2)=1-e^{-\frac x2}$$

这是指数分布的$Exp(\frac12)$的分布函数,原始得证。


\section{5.4.2}

$$P(|\overline{x}-\mu|<1)=P(|\frac{\overline{x}-\mu}{\sqrt{16/n}}|<\frac{1}{\sqrt{16/n}})=2\Phi(\frac{\sqrt{n}}{4})-1\geq 0.95$$

解得$n\geq61.47$,因此n至少为62时,上述不等式成立。

\section{5.4.3}

$$P(|\overline{x}-\overline{y}|>0.2)=P(\frac{|\overline{x}-\overline{y}|}{\sqrt{7/15}}>\frac{0.2}{\sqrt{7/15}})=2(1-\Phi(0.29))=0.7718$$

\section{5.4.8}

X的密度函数为:

$$p_x(x)=\frac{\Gamma(\frac{m+n}{2})(\frac nm)}{\Gamma(\frac n2)\Gamma(\frac m2)}x^{\frac n2 -1}(1+\frac nm x)^{-\frac{m+n}2}$$

因为z在$(0,+\infty)$上单调增,其反函数为:

$$x=\frac{mz}{n(1-z)},\frac{dx}{dz}=\frac m{n(1-z)^2}$$

Z的密度函数为:

$$p_Z(z)=\frac{\Gamma(\frac{m+n}2)}{\Gamma(\frac n2)\Gamma(\frac m2)}z^{\frac n2-1}(1-z)^{\frac m2-1}, \quad 0<z<1$$

因此Z服从$Be(\frac n2, \frac m2)$,两个参数分为为F分布两个自由度的一半。



\section{5.4.11}

根据题意有:

$$c(\overline x-\mu_1)\sim N(0, \frac{c^2\sigma^2}{n}),\quad d(\overline y -\mu_2)\sim N(0, \frac{d^2\sigma^2}{m})$$

$$\frac{(n-1)s_x^2}{\sigma^2}\sim X^2(n-1), \quad \frac{(m-1)s_y^2}{\sigma^2}\sim X^2(m-1)$$

且$\overline x, \overline y, s_x^2, s_y^2$相互独立,故:

$$c(\overline x-\mu_1)+d(\overline y-\mu_2)\sim N(0,\frac{c^2\sigma^2}{n}+\frac{d^2n^2}{m})$$
$$\frac{(n+m-2)s_w^2}{\sigma^2}=\frac{(n-1)s^2_x}{\sigma^2}+\frac{(m-1)s_y^2}{\sigma^2}\sim X^2(n+m-2)$$

因此:

$$t=\frac{c(\overline x-\mu_1)+d(\overline y-\mu_2)}{s_w\sqrt{\frac{c^2}{n}+\frac{d^2}{m}}}=\frac{[c(\overline x-\mu_1)+d(\overline y-\mu_2)]/\sqrt{\frac{c^2\sigma^2}{n}+\frac{d^2\sigma^2}{m}}}{\sqrt{\frac{(n+m-2)s_w^2}{\sigma^2}/(n+m-2)}}$$

即:

$$t\sim t(n+m-2)$$



\section{5.4.14}

$$\frac 1 {\sigma^2}(x_1^2+x_2^2+\cdots+x_{10}^2)\sim X^2(10)$$

$$\frac 1 {\sigma^2}(x_{11}^2+x_{12}^2+\cdots+x_{15}^2)\sim X^2(5)$$

因为两者独立,因此:

$$y=\frac{\frac 1 {\sigma^2}(x_1^2+x_2^2+\cdots+x_{10}^2)/10}{\frac 1 {\sigma^2}(x_{11}^2+x_{12}^2+\cdots+x_{15}^2)/5}=F(10,5)$$

\end{document}





