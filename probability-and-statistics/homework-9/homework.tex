\documentclass{article}
\usepackage{ctex}
\usepackage{geometry}
\usepackage{amsmath}
\usepackage{float}
\usepackage{diagbox}
\geometry{a4paper, scale=0.8}

\title{概率论与数理统计第九次作业}
\author{ZhaohengLi 2017050025}

\begin{document}
\maketitle

\section{3.4.2}
设$X_i$为第i颗骰子出现的点数,$i=1,2,\cdots,n$,那么$X_1,X_2,\cdots,X_n$独立同分布,其分布列为:

\begin{table}[H]
\centering
\begin{tabular}{|l|l|l|l|l|l|l|}
\hline
X & 1 & 2 & 3 & 4 & 5 & 6 \\ \hline
P & 1/6  &  1/6 & 1/6  & 1/6  & 1/6  & 1/6  \\ \hline
\end{tabular}
\end{table}

所以有:

$$E(X_i)=\frac16(1+2+3+4+5+6)=\frac72$$
$$Var(X_i)=\frac16(1^2+2^2+3^2+4^2+5^2+6^2)-\frac16*\frac72=\frac{35}{12}$$

由此可知:

$$E(\sum^n_{i=1}X_i)=\sum^n_{i=1}E(X_i)=\frac72n$$
$$Var(\sum^n_{i=1}X_i)=\sum^n_{i=1}Var(X_i)=\frac{35}{12}n$$


\section{3.4.20}

设:


\begin{equation}
X_i=\left\{
\begin{aligned}
&1,& condition(1,i),\\
&0,& others.
\end{aligned}
\right.
\end{equation}


\begin{equation}
Y_i=\left\{
\begin{aligned}
&1,& condition(6,i),\\
&0,& others.
\end{aligned}
\right.
\end{equation}

$condition(1,i)$和$condition(6,i)$分别代表第i次投掷出现1点和6点。

则有:

$$E(X)=E(Y)=\frac n6$$
$$Var(X)=Var(Y)=\frac{5n}{36}$$

$$XY=\sum^n_{i=1}X_iY_i+2\sum_{i<j}X_iY_j$$

根据实际意义可知,某次投掷时,不可能出现1点同时出现6点,因此有:

$$P(X_iY_i=1)=0$$
$$P(X_iY_i=0)=1-P(X_iY_i=1)=1$$
$$E(X_iY_i)=0$$

当$i,j$不相等时,因为$X_i,Y_j$独立,因此$E(X_iY_j)=E(X_i)E(Y_j)=\frac1{36}$,综上可得:

$$Cov(X,Y)=E(XY)-E(X)E(Y)=2\sum_{i<j}E(X_iY_j)-\frac{n^2}{36}=-\frac{n}{36}$$
$$Corr(X,Y)=\frac{Cov(X,Y)}{\sqrt{Var(X)}\sqrt{Var(Y)}}=-\frac15$$



\section{3.4.44}

(1)

由全概率公式可得:

\begin{equation}
\begin{aligned}
F_Z(z)&=P(XY\leq z)\\
&=P(Y=1)P(XY\leq z|Y=1)+P(Y=-1)P(XY \leq z|Y=-1)\\
&=0.5\Phi(z)+0.5(1-\Phi(-z))\\
&=\Phi(z)
\end{aligned}
\end{equation}

因此$Z\sim N(0,1)$。

(2)

因为$E(X)=0,E(Y)=0$且X与Y相互独立,所以有:

$$Cov(X,Z)=E(X^2)E(Y)-E(X)E(XY)=0$$

因此X与Z不相关。

根据全概率公式有:

\begin{equation}
\begin{aligned}
&P(X\leq 1,XY \geq 1)\\
&=P(Y=1)P(X\leq 1,XY\geq 1| Y=1)+P(Y=-1)P(X\leq 1,XY\geq 1| Y=-1)\\
&=0.5P(X\leq 1,X\geq 1)+0.5P(X\leq 1,-X\geq 1)\\
&=0.5(1-\Phi(1))\\
\end{aligned}
\end{equation}

因为$Z=XY\sim N(0,1)$,所以:

$$P(X\leq 1)P(XY\geq 1)=\Phi(1)(1-\Phi(1))$$

因为$\Phi(1)\neq0.5$,所以$P(X\leq 1, XY\geq1)\neq P(X\leq 1)P(XY\geq1)$,即X与Z不独立。


\section{4.2.5}

特征函数为:

$$\phi (t)=e^{i\mu t -\sigma^2t^2/2}$$

$$E(X)=\frac{\phi'(0)}{i}=\mu$$
$$E(X^2)=\frac{\phi''(0)}{i^2}=\mu^2+\sigma^2$$
$$E(X^3)=\frac{\phi'''(0)}{i^3}=\mu^3+3\mu\sigma^2$$
$$E(X^4)=\frac{\phi''''(0)}{i^4}=\mu^4+6\mu^2\sigma^2+3\sigma^4$$

3阶中心矩和4阶中心矩分别为:

$$E(X-E(X))^3=E(X^3)-3E(X^2)\mu+3E(X)\mu^2-\mu^3=0$$
$$E(X-E(X))^4=E(X^4)-4E(X^3)\mu+6E(X^2)\mu^2-4E(X)\mu^3+\mu^4=3\sigma^4$$
\section{4.2.7}
因为:
$$\phi_X(t)=e^{\lambda_1(e^{it-1})},\quad \phi_Y(t)=e^{\lambda_2(e^{it-1})}$$

所以由独立性可以得到:

$$\phi_{X+Y}(t)=\phi_X(t)\phi_Y(t)=e^{(\lambda_1+\lambda_2)(e^{it-1})}$$

这是泊松分布$P(\lambda_1+\lambda_2)$的特征函数,由唯一性定理可以知道:

$$X+Y\sim P(\lambda_1+\lambda_2)$$
\section{4.2.10}
因为:
$$\phi_{X_i}(t)=(1-\frac{it}{\lambda})^{-1}$$

所以由独立性可以得到:

$$\phi_{Y_n}(t)=(1-\frac{it}{\lambda})^{-n}$$

这是伽马分布$Ga(n,lambda)$的特征函数,由唯一性定理可以知道:

$$Y_n\sim Ga(n,\lambda)$$

\section{4.2.13}

因为$X_j$的特征函数为:

$$\phi_j(t)=e^{i\mu t-\sigma^2t^2/2}$$

所以由独立性可以得到:

$$\phi_{\overline X}(t)=[\phi_i(t/n)]^n=e^{i\mu t-\sigma^2t^2/(2n)}$$

这是正态分布$N(\mu, \sigma^2/n)$的特征函数,由唯一性定理可以知道:

$$\overline X=\frac1n\sum^n_{i=1}X_i \sim N(\mu, \frac{\sigma^2}{n})$$
\end{document}





