\documentclass{article}
\usepackage{ctex}
\usepackage{geometry}
\usepackage{amsmath}
\usepackage{float}
\usepackage{diagbox}
\geometry{a4paper, scale=0.8}

\title{概率论与数理统计第八次作业}
\author{ZhaohengLi 2017050025}

\begin{document}
\maketitle

\section{3.3.6}
(1)

因为$x,y>0$,所以当$z\leq 0$时,$F_Z(z)=0$。当$z>0$时,有下式:

$$F_Z(z)=P(Z\leq z)=P(X+Y\leq 2z)=\int^{2x}_0\int^{2z-x}_0e^{-(x+y)}dydx=1-e^{-2x}-2ze^{-2x}$$

所以,当$z\leq0$时,有$p_Z(z)=0$,当$z>0$时,有$p_Z(z)=4ze^{-2z}$。

(2)

当$z\leq 0$时,

$$F_Z(z)=P(Z\leq z)=P(X-Y\leq z)=\int^{+\infty}_0\int^{+\infty}_{y-z}e^{-(x+y)}dydx=0.5e^{z}$$
$$p_Z(z)=0.5e^{z}$$

当$z>0$时,

$$F_Z(z)=P(Z\leq z)=P(X-Y\leq z)=\int^{+\infty}_0\int^{x+z}_{0}e^{-(x+y)}dydx=0.5e^{-z}$$
$$p_Z(z)=0.5e^{-z}$$

\section{3.3.9}
首先可以得到:

$$p_Z(z)=\int^{+\infty}_{-\infty}p_X(x)p_Y(z-x)dx$$

(1)

\begin{equation}
p_Z(z)=\left\{
\begin{aligned}
&\int^z_{0}dx=z,&0\leq z<1,\\
&\int^{1}_{z-1}dx=2-z,&1\leq z<2,\\
&0,&others.
\end{aligned}
\right.
\end{equation}

(2)
在$\{0\leq x \leq 1\}$与$\{z-x\geq 0\}$的交集区域中求得密度函数:

\begin{equation}
p_Z(z)=\left\{
\begin{aligned}
&\int^z_{0}e^{-(z-x)}dx=1-e^{-z},&0\leq z<1,\\
&\int^{1}_{0}e^{-(z-x)}dx=e^{-z}(e-1),&z>1,\\
&0,&others.
\end{aligned}
\right.
\end{equation}

\section{3.3.10}

(1)
因为当$0<x<1$时,$p_X(x)=1$,且当$y>0$时,$p_Y(y)=e^{-y}$。所以$Z=X/Y$的密度函数可以通过如下计算获得:

$$p_Z(z)=\int^{1/z}_0e^{-y}ydy=1-(1+\frac1z)e^{-1/z}$$

(2)
与问题(1)方法类似,只是密度函数和积分区间不同。

$$p_Z(z)=\int^{+\infty}_{-\infty}\lambda_1\lambda_2e^{-\lambda_1zy}e^{-\lambda_2y}ydy=\frac{\lambda_1\lambda_2}{(\lambda_1z+\lambda_2)^2}$$



\section{3.3.11}

记$X_(3)=max\{X_1,X_2,X_3\},X_{(1)}=min\{X_1,X_2,X_3\},X_{(2)}=X_1,X_2,X_3$三者中取值处于中间的。

$$p_{(X_{(1)},X_{(2)},X_{(3)})}(x_{(1)},x_{(2)},x_{(3)})=6,\quad0<x_{(1)},x_{(2)},x_{(3)}<1$$

因此所求概率为

$$P(X_{(3)}\geq X_{(1)}+X_{(2)})=6\int^1_0\int^{x_{(3)}}_0\int^{min|x_{(3)}-x_{(2)},x_{(2)}|}_0dx_{(1)}dx_{(2)}dx_{(3)}$$

$$=6\int^1_0\int^{x_{(3)}}_0\frac{x_{(3)}-|x_{(3)}-2x_{(2)}|}{2}dx_{(2)}dx_{(3)}$$

$$=6\int_0^1\frac18x^2_{(3)}dx_{(3)}+6\int_0^1\frac18x_{(3)}^2dx_{(3)}=0.5$$


\section{3.3.15}

(X,Y)的联合分布函数为:


\begin{equation}
p_{X,Y}(x,y)=\left\{
\begin{aligned}
&\frac1\pi,&0< x^2+y^2<1,\\
&0,&others.
\end{aligned}
\right.
\end{equation}

$$P_{R,\theta}(r,\theta)=P_{X,Y}(x(r,\theta).y(r,\theta)) |\frac{1}{\sqrt{x^2+y^2}}|=\frac r\pi$$

$$0<r<1,0<\theta<2\pi$$


\section{3.3.16}

(1)
首先求出反函数变换的雅可比行列式:
$$J=-uv-u(1-v)=-u$$
当(U,V)取值在$\{u>0,0<v<1\}$内有:

$$P_{U,V}(u,v)=p_X(uv)p_Y(u(1-v))|-u|=e^{-uv}e^{-u(1-v)}=ue^{-u}$$

(2)
$$p_U(u)=\int^{+\infty}_{-\infty}p_{U,V}(u,v)dv=\int^{1}_0ue^{-u}dv=ue^{-u},\quad u>0$$

$$p_V(v)=\int^{+\infty}_{-\infty}p_{U,V}(u,v)du=\int^{1}_0ue^{-u}du=1,\quad 0<v<1$$


因为$p_{U,V}(u,v)=p_U(u)p_V(v)$,因此U和V独立。


\section{3.4.4}
n个点把区间(0,1)区间分成n+1段,他们的长度依次记为$Y_1,Y_2,\cdots,Y_{n+1}$。因为n点随机取得,所以$Y_1,Y_2,\cdots,Y_{n+1}$具有相同的分布,从而具有相同的数学期望,此外,$Y_1+Y_2+\cdots+Y_{n+1}=1$,因此:

$$E(Y_1)=E(Y_2)=\cdots=E(Y_{n+1})=\frac1{n+1}$$

而距离最远的亮点间距离为$Y_2+Y_3+\cdots+Y_{n}$,因此所求期望为:

$$E(Y_2+Y_3+\cdots+Y_{n})=\frac{n-1}{n+1}$$


\section{3.4.9}


\begin{equation}
\begin{aligned}
E|X-Y|=&\sum^m_{i=1}\sum^m_{j-1}|i-j|\frac{1}{m^2}=\frac1{m^2}\sum^m_{i=1}(\sum^i_{j=1}(i-j)+\sum^m_{j=i+1}(j-i))\\
=&\frac1{m^2}\sum^m_{i=1}(i^2-i-mi=\frac{m^2}2+\frac m2)\\
=&\frac{(m-1)(m+1)}{3m}
\end{aligned}
\end{equation}


\section{3.4.24}

$$Var(U)=Var(2X+Y)=4Var(X)+Var(Y)=5\lambda$$

$$Var(V)=Var(2X-Y)=4Var(X)+Var(Y)=5\lambda$$

所以

$$Cov(U,V)=Cov(2X+Y,2X-Y)=4Var(X)-Var(Y)=3\lambda$$

由此得出:

$$Corr(U,V)=\frac{Cov(U,V)}{\sqrt{Var(U)}\sqrt{Var(V)}}=\frac35$$
\section{3.4.27}
$$E(X)=\int^1_0\int^x_{-x}xdydx=\frac23$$
$$E(Y)=\int_0^1\int^x_{-x}ydydx=0$$

$$E(XY)=\int^1_0\int^x_{-x}xydydx=0$$
$$Cov(X,Y)=E(XY)-E(X)E(Y)=0$$

由下列式子可以知道XY不独立,故协方差为0.
\begin{equation}
p(x)=\left\{
\begin{aligned}
&2x,&0<x<1,\\
&0,&others
\end{aligned}
\right.
\end{equation}

\begin{equation}
p(y)=\left\{
\begin{aligned}
&1+y,&-1<y<0,\\
&1-y,&0<y<1,\\
&0,&others
\end{aligned}
\right.
\end{equation}

\section{3.4.32}

(1)

$$E[max\{X,Y\}]=E[\frac12(X+Y+|X-Y|)]=\frac12E|X-Y|$$
因为$X-Y\sim N(0,2(1-\rho))$,所以:
$$E[max\{X,Y\}]=\frac{1}{2\sqrt{2\pi}\sqrt{2(1-\rho)}}\int^{+\infty}_{-\infty}|x|exp\{-\frac{x^2}{4(1-\rho)}\}dx=\sqrt{\frac{1-\rho}{\pi}}$$

(2)

$$Cov(X-Y,XY)=Cov(X,XY)-Cov(Y,XY)=E(X^2Y)-E(X)E(XY)-E(Y^2X)+E(Y)E(XY)$$
因为$E(X)=E(Y)=0$,所以:

$$Cov(X-Y,XY)=E(X^2Y)-E(XY^2)$$


因为对称性,所以$E(X^2Y)=E(XY^2)$,因此:

$$Cov(X-Y,XY)=0,\quad Corr(X-Y,XY)=0$$

这说明,当$(X,Y)\sim N(0,0,1,1,\rho)$时,$X-Y$与$XY$不相关。


\section{3.4.35}

$(X,Y)$的联合密度函数为:

\begin{equation}
p_{X,Y}(x,y)=\left\{
\begin{aligned}
&\frac12,&(x,y)\in G,\\
&0,&others
\end{aligned}
\right.
\end{equation}

$$P(U=0)=\int^1_0dy\int^y_0\frac12dx=\frac14$$
$$P(U=1)=1-P(U=0)=\frac34$$


$$P(V=0)=\int^2_0dx\int^{\frac x2}_0\frac12dy=\frac12$$
$$P(V=1)=1-P(V=0)=\frac12$$

$$Var(U)=\frac34(1-\frac34)=\frac3{16}$$
$$Var(V)=\frac12(1-\frac12)=\frac14$$

$$E(UV)=P(UV=1)=P(U=1,V=1)=P(X>Y,X>2Y)=P(X>2Y)=\frac12$$

$$Cov(U,V)=\frac12-\frac34*\frac12=\frac18$$
所以相关系数计算为:


$$Corr(U,V)=\frac{Cov(U,V)}{\sqrt{Var(U)}\sqrt{Var(V)}}=\frac1{\sqrt 3}=0.5774$$
\end{document}





