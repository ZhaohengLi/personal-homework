\documentclass{article}
\usepackage{ctex}
\usepackage{geometry}
\usepackage{amsmath}
\usepackage{float}
\usepackage{diagbox}
\geometry{a4paper, scale=0.8}

\title{概率论与数理统计第十次作业}
\author{ZhaohengLi 2017050025}

\begin{document}
\maketitle

\section{4.1.5}

充分性证明:

令$f(x)=\frac{x}{1+x}, x>0$,则有$f'(x)=\frac{x}{(1+x)^2}>0, x>0$,故$f(x)$是$x$的严格单调增函数,因而对任意的$\epsilon>0$,有:

$$\{|X_n-X|>\epsilon\} \subset \{\frac{|X_n-X|}{1+|X_N-X|}>\frac{\epsilon}{1+\epsilon}\}$$

于是对于任意的$\epsilon > 0$,当$n \rightarrow \infty$时,有:

$$P\{|X_n-X|>\epsilon\} \leq P\{\frac{|X_n-X|}{1+|X_N-X|}>\frac{\epsilon}{1+\epsilon}\} \leq \frac{1+\epsilon}{\epsilon}(\frac{|X_n-X|}{1+|X_N-X|})\rightarrow 0$$

必要性证明:

对于任意的$\epsilon >0$,令$A_{\epsilon}=\{|X_n-X|>\epsilon\}$,因为$X_n\rightarrow X$,故存在充分大的N,使得当$n\geq N$时,有$P(A_{\epsilon})<\epsilon$,于是有:

$$E(\frac{|X_n-X|}{1+|X_N-X|})=E(\frac{|X_n-X|}{1+|X_N-X|}I_{A_{\epsilon}})+E(\frac{|X_n-X|}{1+|X_N-X|}I_{\overline{A_{\epsilon}}})$$

$$E(\frac{|X_n-X|}{1+|X_N-X|})\leq P(A_\epsilon)+\epsilon<2\epsilon$$

当$n\rightarrow \infty$时,有:

$$E(\frac{|X_n-X|}{1+|X_N-X|})\rightarrow 0$$


\section{4.1.9}
$$P(X_n+Y_n\leq z)=P(X_n+Y_n \ leq z-a+\epsilon)+lim_{n\rightarrow \infty}P(|Y-a|>\epsilon)$$
$$P(X_n+Y_n\leq z)\leq P(X_n\leq z-a+\epsilon)+P(|Y_n-a|>\epsilon)$$

因此

$$lim_{n\rightarrow \infty}P(X_n+Y_n\leq z)\leq lim_{n\rightarrow \infty}P(X_n\leq z-a+\epsilon)+lim_{n\rightarrow \infty}P(|Y_n-a|>\epsilon)$$

$$lim_{n\rightarrow \infty}P(X_n+Y_n\leq z)=F(z-a+\epsilon)$$

因为:

$$P(X_n+Y_n\leq z) \geq P(X_n\leq z-a+\epsilon)+P(|Y_n-a|>\epsilon)$$


所以有:
$$lim_{n\rightarrow \infty}P(X_n+Y_n\leq z) \geq P(X\leq z-a-\epsilon)=F(z-a-\epsilon)$$

由上述两个关系式,可以得出:

$$lim_{n\rightarrow \infty}P(X_n+Y_n\leq z)=F(z-a)$$

\section{4.1.13}
当$x<0$时,有$P(Y_n \leq x)=0$。

当$x<\beta$时,有$P(Y_n \leq x)=1$。

当$0\leq x < \beta$时,有:

$$P(Y_n \leq x) = \prod^n_{i=1}P(X_i\leq x)=\prod^n_{i=1}\int^x_0\frac1\beta dx=(\frac x\beta)^n$$

所以,对于任意的$\epsilon > 0 (\epsilon < \beta)$,当$n\rightarrow \infty$时,有:

$$P(|Y_n-\beta|\geq \epsilon)=P(Y_n \leq \beta - \epsilon)=(\frac{\beta-\epsilon}{\beta})^n \rightarrow 0$$

因此结论得证。

\section{4.1.15}
$$LnY_n=\frac1n\sum^n_{i=1}X_i$$
$$E(lnX_i)=\int^i_0lnxdx=-1$$
$$E(lnX_i)^2=\int^1_0(lnx)^2dx=2$$
$$Var(lnX_i)=1$$

由切比雪夫不等式可以得到:

对任意的$\epsilon > 0$,有:

$$P(|\frac1n\sum^n_{i=1}lnX_i-1(-1)|\geq\epsilon)\leq \frac1{n\epsilon^2}\rightarrow 0$$

因此 $lnY_n\rightarrow -1$,故$Y_n=(\prod^n_{i=1}X_i)^{\frac1n}\rightarrow^{-1}$,即$c=e^{-1}$。


\section{4.1.19}
假设$E(X_n)=0$,$\sigma^2,S_n^2$保持不变。

有:

$$S_n^2=\frac1n\sum^n_{k=1}(X_k-\overline{X}_n)^2=\frac1n\sum^n_{k=1}X_k^2-(\overline{X}_n)^2$$

因为:

$$\overline{X}_n=\frac1n\sum^n_{i=1}X_i\rightarrow\frac1n\sum^n_{i=1}E(X_i)=0$$

$$\frac1n\sum^n_{k=1}X^2_k\rightarrow \sigma^2$$

因此有:

$$S^2_n=\frac1n\sum^n_{k=1}X^2_k-(\overline{X}_n)^2\rightarrow \sigma^2$$

\section{4.3.3}

因为随机变量相互独立,而且有$E(X_n)=0,Var(X_n)=2,n=2,3,\cdots.$,因此有马尔可夫条件:

$$\frac1{n^2}Var(\sum^n_{i=2}X_i)=\frac{2(n-1)}{n^2}\rightarrow 0,\quad (n\rightarrow \infty)$$

由马尔可夫大数定律可以知道 ${X_n}$服从大数定律。

\section{4.3.5}

因为$E(X_n)=p_n, Var(X_n)=p_n(1-p_n)\leq\frac14$,所以由变量的独立性可以得到:

$$\frac1{n^2}Var(\sum^n_{k=2}X_k)=\frac{1}{4n}\rightarrow 0,\quad (n\rightarrow \infty)$$


由马尔可夫大数定律可以知道 ${X_n}$服从大数定律。


\section{4.3.9}
因为:

$$\frac1{n^2}Var(\sum^n_{i=1}X_i)=\frac{n\sqrt{n}}{n^2}\rightarrow 0,\quad (n\rightarrow \infty)$$


由马尔可夫大数定律可以知道 ${X_n}$服从大数定律。



\section{4.3.13}

充分性证明:

对于任意的$\epsilon > 0$,当$t>0$时,$f(t)=\frac{t^2}{1+t^2}$是增函数。

因此当$|y-a_n|\geq\epsilon$时,有:

$$\frac{|y-a_n|^2}{1+|y-a_n|^2}\geq \frac{\epsilon^2}{1+\epsilon^2}$$

$$\frac{1+\epsilon^2}{\epsilon^2}\frac{|y-a_n|^2}{1+|y-a_n|^2}\geq1$$

因此有:

$$P(|Y_n-a_n|\geq \epsilon)=\int_{|y-a_n|\geq\epsilon}dF_{Y_n}(y)\leq\frac{1+\epsilon^2}{\epsilon^2}E[\frac{(Y_n-a_n)^2}{1+(Y_n-a_n)^2}]$$

所以当$lim_{n\rightarrow+\infty}E[\frac{(Y_n-a_n)^2}{1+(Y_n-a_n)^2}]=0$时,有$lim_{n\rightarrow+\infty}P(|Y_n-a_n|\geq\epsilon)=0$,因此服从大数定律。

必要性证明:

设${X_n}$服从大数定律,即$lim_{n\rightarrow+\infty}P(|Y_n-a_n|\geq\epsilon)=0$,那么对于$\epsilon > 0$,存在N,当$n>N$时,有:

$$P(|Y_n-a_n|\geq\epsilon)\leq\epsilon$$

因为函数$f(t)$为增函数,且$0<f(t)<1$,那么:

$$0\leq E[\frac{(Y_n-a_n)^2}{1+(Y_n-a_n)^2}]\leq\epsilon^2+\epsilon$$

因此可以得到:
$$lim_{n\rightarrow \infty}E[\frac{(Y_n-a_n)^2}{1+(Y_n-a_n)^2}]=0$$


\end{document}





