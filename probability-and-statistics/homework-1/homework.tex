\documentclass{article}
\usepackage{ctex}

\title{概率论与数理统计第一次作业}

\author{ZhaohengLi 2017050025}

\begin{document}
\maketitle

% --------------------
\section{第5题}
% --------------------
由题意可知$\Omega=\{(B,C)|B,C=1,2...,6\}$,当$B^{2}-4C>=0$时方程有实根,当$B^{2}-4C=0$时方程有重根。\\
$p=P(B^{2}>=4C)=\frac{19}{36}$\\
$q=P(B^{2}=4C)=\frac{2}{36}=\frac{1}{18}$\\

% --------------------
\section{第6题}
% --------------------
(1)$P=\frac{C^{4}_{13}}{C^{4}_{52}}=0.0026$\\
(2)$P=\frac{4*C^{4}_{13}}{C^{4}_{52}}=0.0105$\\
(3)$P=\frac{13^{4}}{C^{4}_{52}}=0.1054$\\
(4)$P=\frac{2*C^{4}_{26}}{C^{4}_{52}}=0.1104$\\
% --------------------
\section{第9题}
% --------------------
$P=\frac{5*4+3*6}{8*10}=\frac{19}{40}$\\
% --------------------
\section{第10题}
% --------------------
$P=\frac{2*C^{1}_{k-1}*C^{1}_{n-k}}{C^{2}_{n}}$

\section{第21题}
$P=\frac{C^{3}_{12}*3^{9}}{3^{12}}=0.212$


% --------------------
\section{第22题}
% --------------------
利用插棍思想,将盒子抽象为两个木棍之间的空间,使用N+1根木棍组成N个盒子,放入问题即变为排列问题。\\利用01序列来表示各种情况,例如“1001011”表示第1个盒子有2个球,第2个盒子有1个球,第3个盒子没有球。\\
样本总量为
\begin{equation}
    C^{n}_{N-1+n}
\end{equation}
(1)此时不妨假设指定的盒子为第1个盒子,问题化简为n-k个球放入N-1个盒子里\\
\begin{equation}
    P=\frac{C^{n-k}_{N-2-k+n}}{C^{n}_{N-1+n}}\quad(0<=k<=n)
\end{equation}{}

(2)首先从N个盒子中选出m个盒子作为空盒,再将n个球放入剩余的N-m个盒子中,为保证每个盒子中至少有一个球。我们可以排列球在球与球之间插入木棍。\\
\begin{equation}
    P=\frac{C^{m}_{N}*C^{N-m-1}_{n-1}}{C^{n}_{N-1+n}}\quad(N-n<=m<=N-1)
\end{equation}
(3)将整体分为两部分分析,得到如下\\
\begin{equation}
    P=\frac{C^{m-1}_{m+j-1}*C^{n-j}_{N+n-m-j-1}}{C^{n}_{N-1+n}}\quad(1<=m<=N, 0<=j<=n)
\end{equation}

\section{第16题}
\begin{equation}
   P(AB)=P(\overline{A}\cap\overline{B})=P(\overline{A\cup B})=1-P(A\cup B)=1-P(A)-P(B)+P(AB)
\end{equation}
因为$P(A)=p$所以$P(B)=1-p$。

\section{第19题}
(1)$P(A)>=P(A(B\cup C))=P(AB\cup AC)=P(AB)+P(AC)-P(ABC)>=P(A)+P(B)-P(BC)$\\
(2)因为$P(A\cup B\cup C)=P(A)+P(B)+P(C)-P(AB)-P(BC)-P(AC)+P(ABC)<=1$,\\
所以$P(A)+P(B)+P(C)-1$\\
$<=P(AB)+P(BC)+P(AC)-P(ABC)$\\
$<=P(AB)+P(BC)+P(AC)$

\section{第21题}
因为$P(ABC)=P(\overline{A}\cap\overline{B}\cap\overline{C})=1-P(A\cup B\cup C)$\\
$=1-P(A)-P(B)-P(C)+P(AB)+P(BC)+P(AC)-P(ABC)$\\
$=P(AB)+P(BC)+P(AC)-\frac{1}{2}-P(ABC)$\\
所以原式成立。
\end{document}