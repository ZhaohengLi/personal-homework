\documentclass{article}
\usepackage{ctex}
\usepackage{geometry}
\usepackage{amsmath}
\usepackage{float}
\usepackage{diagbox}
\geometry{a4paper, scale=0.8}

\title{概率论与数理统计第七次作业}
\author{ZhaohengLi 2017050025}

\begin{document}
\maketitle

\section{3.1.7}
(1)
$$P(0<X<0.5,0.25<Y<1)=4\int^{0.5}_0xdx\int^1_{0.25}ydy=\frac{15}{64}$$

(2)
$$P(X=Y)=0$$

(3)
$$P(X<Y)=4\int^1_0\int^y_0xydxdy=4\int_0^1\frac12y^3dy=0.5$$

(4)
\begin{equation}
F(x,y)=\left\{
\begin{aligned}
&0,&\quad x<0 \quad or \quad y<0,\\
&x^2y^2,&\quad0\leq x <1,0\leq y<1,\\
&x^2,&\quad 0\leq x < 1, 1 \leq y,\\
&y^2,&\quad 1\leq x,0\leq y<1,\\
&1,&\quad x\geq1,y\geq1.
\end{aligned}
\right.
\end{equation}
\section{3.1.8}
设$D={(x,y)|-1\leq x,y \leq 1},G={(x,y)|x^2+y^2\leq 1}$。因为二维随机变量服从D上的均匀分布,且D的面积$S_D$为4,G的面积$S_G$为$\pi$,所以:

$$P(X^2+Y^2\leq 1)=\frac{S_G}{S_D}=\frac{\pi}{4}$$


\section{3.1.10}
解决此类问题的首先步骤是画出图像,找出重叠部分面积,在根据面积确定积分区域,积分上下限,由于LATEX画图十分繁琐,在此处便不在将草稿图呈现,望助教谅解。
(1)
$$P(X>0.5,Y>0.5)=6\int^1_{0.5}\int^y_{0.5}(1-y)dxdy=\frac18$$

(2)
$$P(X<0.5)=6\int^{0.5}_{0}\int^{1}_{x}(1-y)dydx=\frac78$$


$$P(Y<0.5)=6\int^{0.5}_{0}\int^{0.5}_{x}(1-y)dydx=\frac12$$

(3)
$$P(X+Y<1)=6\int^{0.5}_{0}\int^{1-x}_{x}(1-y)dydx=\frac34$$

\section{3.1.13}
解决此类问题的首先步骤是画出图像,找出重叠部分面积,在根据面积确定积分区域,积分上下限,由于LATEX画图十分繁琐,在此处便不在将草稿图呈现,望助教谅解。

$$P(X+Y\leq1)=\int^{0.5}_{0}\int^{1-x}_{x}e^{-y}dydx=1+e^{-1}-2e^{-0.5}=0.1548$$



\section{3.2.5}

(1)
\begin{equation}
p_X(x)=\left\{
\begin{aligned}
&\int^{+\infty}_{0}e^{-y}dy=e^{-x},&x>0,\\
&0,&others.
\end{aligned}
\right.
\end{equation}

\begin{equation}
p_Y(y)=\left\{
\begin{aligned}
&\int^{y}_{0}e^{-y}dy=ye^{-y},&y>0,\\
&0,&others.
\end{aligned}
\right.
\end{equation}

(2)
该分布区域为曲线与x轴包围的拱形区域,
\begin{equation}
p_X(x)=\left\{
\begin{aligned}
&\int^{1-x^2}_{0}\frac54(x^2+y)dy=\frac58(1-x^4),&-1<x<1,\\
&0,&others.
\end{aligned}
\right.
\end{equation}

\begin{equation}
p_Y(y)=\left\{
\begin{aligned}
&\int^{\sqrt{1-y}}_{-\sqrt{1-y}}\frac54(x^2+y)dx=\frac56\sqrt{1-y}(1+2y),&0<y<1,\\
&0,&others.
\end{aligned}
\right.
\end{equation}

(3)
\begin{equation}
p_X(x)=\left\{
\begin{aligned}
&\int^{x}_{0}\frac1xdy=1,&0<x<1,\\
&0,&others.
\end{aligned}
\right.
\end{equation}

\begin{equation}
p_Y(y)=\left\{
\begin{aligned}
&\int^{1}_{y}\frac1xdx=-lny,&0<y<1,\\
&0,&others.
\end{aligned}
\right.
\end{equation}

\section{3.2.13}
(1)
该分布区域为X轴上方的倒三角区域,由于LATEX画图十分繁琐,在此处便不在将草稿图呈现,望助教谅解。

对x分区间讨论,

当$-1<x<0$时,有$p_X(x)=\int^1_{-x}dy=1+x$,当$0<x<1$时,有$p_X(x)=\int^1_{x}dy=1-x$,因此有:

\begin{equation}
p_X(x)=\left\{
\begin{aligned}
&1+x,&-1<x<0,\\
&1-x,&0<x<1,\\
&0,&others.
\end{aligned}
\right.
\end{equation}

当$0<y<1$时,有:

\begin{equation}
p_Y(y)=\left\{
\begin{aligned}
&\int^{y}_{-y}dx=2y,&0<y<1,\\
&0,&others.
\end{aligned}
\right.
\end{equation}

(2)
因为$p(x,y)\neq p_X(x)p_Y(y) $,所以XY不独立。

\section{3.3.1}
$$P(U=1)=P(X=0,Y=1)+P(X=1,Y=1)=0.12$$
$$P(U=2)=P(X=2,Y=1)+\sum^2_{i=0}P(X=i,Y=2)=0.37$$
$$P(U=3)=\sum^3_{i=0}P(X=i,Y=3)=0.51$$
\begin{table}[H]
\centering
\begin{tabular}{|l|l|l|l|}
\hline
U & 1    & 2    & 3    \\ \hline
P & 0.12 & 0.37 & 0.51 \\ \hline
\end{tabular}
\end{table}

$$P(V=0)=\sum^3_{i=0}P(X=0,Y=i)=0.40$$
$$P(V=1)=P(X=2,Y=1)+\sum^3_{i=0}P(X=1,Y=i)=0.44$$
$$P(V=2)=P(X=2,Y=2)+P(X=2,Y=3)=0.16$$


\begin{table}[H]
\centering
\begin{tabular}{|l|l|l|l|}
\hline
V & 0    & 1    & 2    \\ \hline
P & 0.40 & 0.44 & 0.16 \\ \hline
\end{tabular}
\end{table}

\section{3.3.4}
(1)
$$P(Z=0)=P(X=0,Y=0)=P(X=0)P(Y=0)=0.5*0.5=0.25$$
$$P(Z=1)=1-P(Z=0)=0.75$$

(2)
$$P(X\leq i)=\sum^i_{j=i}(1-p)^{j-1}p=p\frac{1-(1-p)^i}{1-(1-p)}=1-(1-p)^i,\quad i=1,2,\cdots.$$

$$P(Z=i)=P(Z\leq i)-P(Z\leq i-1)=P(X\leq i)P(Y\leq i)-P(X\leq i-1)P(Y\leq i-1)$$

$$P(Z=i)=(1-p)^{i-1}p[2-(1-p)^{i-1}-(1-p)^i],\quad i=1,2\cdots.$$

\section{3.3.7}

$$F_Z(z)=P(Z\leq z)=P(X-Y\leq z)=\int^z_0\int^x_0 3xdydx+\int^1_z\int^x_{x-z} 3xdydx$$
$$F_Z(z)=\frac32z-\frac12z^3$$

$$p_Z(z)=F'_Z(z)=\frac32(1-z^2),\quad 0<z<1$$

另外在区间(0,1)外的z有$p_Z(z)=0$。

\section{3.3.12}
$$Z=max\{X_1,X_2\}-min\{X_1,X_2\}=\frac{X_1+X_2+|X_1-X_2|}{2}-\frac{X_1+X_2-|X_1-X_2|}{2}$$

$$Z=max\{X_1,X_2\}-min\{X_1,X_2\}=|X_1-X_2|$$

设$U=X_1-X_2,V=X_2$,则当$0<u+v<1,0<v<1$时,有下式:

$$P_{U,V}(u,v)=p_{X_1}(u+v)p_{X_2}(v)=2(u+v)*2v$$

因此得到密度函数为

\begin{equation}
p_U(u)=\left\{
\begin{aligned}
&\int^1_{-u}4(u+v)vdv=-\frac23u^3+2u+\frac43,&-1<u<0,\\
&\int^{1-u}_{0}4(u+v)vdv=\frac23u^3-2u+\frac43,&0<u<1,\\
&0,&others.
\end{aligned}
\right.
\end{equation}

因为当$0<z<1$时,$Z=|U|$的分布函数为:

$$F_Z(z)=P(|U|\leq z)=P(-z\leq U\leq z)=F_U(z)-F_U(-z)$$

所以Z的密度函数为:

$$p_Z(z)=p_U(z)+p_U(-z)=\frac43z^3-4z+\frac83,\quad 0<z<1.$$








\end{document}


