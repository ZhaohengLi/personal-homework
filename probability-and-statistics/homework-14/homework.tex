\documentclass{article}
\usepackage{ctex}
\usepackage{float}
\usepackage{geometry}
\usepackage{amsmath}
\usepackage{graphicx}
\geometry{a4paper, scale=0.8}

\title{概率论与数理统计第十四次作业}
\author{ZhaohengLi 2017050025}

\begin{document}
\maketitle

\section{6.4.1}

$$MSE(\hat g)=E(\hat g-\theta)^2=E(\hat g - \overline g)^2+MSE(\overline g)+2E[(\hat g-\overline g)(\overline g-\theta)]$$

因为:

$$\overline g = E(\hat g | T)$$

所以有:

$$E[(\hat g- \overline g)|T]=0$$

$$E[(\hat g-\overline g)(\overline g-\theta)]=0$$

因此:

$$MSE(\hat g)=E(\hat g - \overline g)^2+MSE(\overline g)\geq MSE(\overline g)$$

\section{6.4.2}
由题意可以知道:

$$E(aT_1+bT_2)=a\theta_1+b\theta_2$$
$$Cov(aT_1+bT_2,\phi)=aCov(T_1,\phi)+bCov(T_2,\phi)=0$$

因此$aT_1+bT_2$是$a\theta_1+b\theta_2$的UMVUE。
\section{6.4.7}

因为:

$$lnp(x:\theta)=ln2+ln\theta-3lnx-\theta/x^2$$

所以有:

$$\frac{\partial lnp(x;\theta)}{\partial\theta}=\frac1\theta-\frac1{x^2},\frac{\partial^2 lnp(x;\theta)}{\partial^2\theta}=\frac1{\theta^2}$$

由此得到:

$$I(\theta)=-E(\frac{\partial^2 lnp(x;\theta)}{\partial^2\theta})=\frac1{\theta^2}$$

\section{6.4.14}
(1)

$x_1$的密度函数表示为:

$$p(x,\theta)=(\frac{1-\theta}{2})^{\frac12(x^2-x)}(\frac12)^{1-x^2}(\frac\theta2)^{\frac12(x^2+x)}$$

因此相应的对数似然函数解得:

$$\hat \theta = \frac12+\frac{\sum_{i=1}^n x_i}{2\sum^n_{i=1}x_i^2}$$

所以:

$$\sum^n_{i=2}x_i^2\sim b(n-1,\frac12)$$

$$E\hat \theta_1=\frac12+\frac n2E(\frac{x_1}{x_1^2+\sum_{i=2}^n x_i^2})=\frac12+\frac12(\theta-\frac12)(1-\frac1{2^n})$$

因为:

$$n\frac1{2^n}(\frac11(C^0_{n-1})+\frac12(c^1_{n-1})+\cdots+\frac1n(\frac{n-1}{n-1}))=1-\frac1{2^n}$$

所以$\hat \theta_1=\frac12+\frac{\sum^n_{i=1}x_i}{2\sum^n_{i=1}x_i^2}$不是$\theta$的无偏估计。

\section{6.6.2}
由已知条件得到$\mu$的0.95置信区间为:

$$[\overline x - u_{1-\alpha/2}\alpha/\sqrt{n}, \overline x + u_{1-\alpha/2}\alpha/\sqrt{n}]$$

解:

$$n\geq (2/k)^2\sigma^2u^2_{1-\alpha/2}$$

得到n至少为$(\frac{3.92\sigma}{k})^2$,才能保证题目要求。



\section{6.6.3}
(1)

Y=lnX的样本值为

$$-0.6931,0.2231,-0.23231,0.6931$$

置信区间为:

$$[\overline x - u_{1-\alpha/2}\alpha/\sqrt{n}, \overline x + u_{1-\alpha/2}\alpha/\sqrt{n}]=[-0.9800,0.9800]$$

(2)
因为是严格增函数,因此可以利用(1)的结果:

$$[e^{-0.98+0.5},e^{0.98+0.5}]=[0.6188,4.3929]$$



\section{6.6.7}
由中心极限定理可以知道,当n较大时,样本均值$\overline x \sim N(\lambda,\frac\lambda n)$,因而$u=\frac{\overline x -\lambda}{\sqrt{\lambda/n}}\sim N(0,1)$,因此有:

$$P(|\frac{\overline x -\lambda}{\sqrt{\lambda/n}}|\leq u_{1-\alpha/2})=1-\alpha$$

因为括号里面的事件相当于$(\overline x-\lambda )^2\leq u^2_{1-\alpha/2}\lambda/n$,因而得到:

$$\lambda^2-(2\overline x+\frac1nu^2_{1-\alpha/2})\lambda+\overline x^2\leq 0$$

二次曲线与$\lambda$轴有两个交点,记为$\lambda_L,\lambda_U,(\lambda_L<\lambda_U)$,则有$P(\lambda_L\leq\lambda\leq\lambda_U)=1-\alpha$,其中$\lambda_L,\lambda_U$可以表示为:

$$\frac{2\overline x+\frac1nu^2_{1-\alpha/2}\pm \sqrt{(2\overline x+\frac1nu^2_{1-\alpha/2})^2-4\overline x^2}}{2}$$

因此题目得证。

\section{6.6.9}

(1)

$$[\overline x-\overline y - u_{1-\alpha/2} \sqrt{\frac{\sigma_1^2}{n_1} + \frac{\sigma_2^2}{n_2}},\overline x-\overline y +u_{1-\alpha/2} \sqrt{\frac{\sigma_1^2}{n_1} + \frac{\sigma_2^2}{n_2}}]$$
计算得到:

$$[-0.939,12.0939]$$

(2)

$$[\overline x-\overline y-\sqrt{\frac{n_1+n_2}{n_1n_2}}s_wt_{1-\alpha/2}(n_1+n_2-2),\overline x-\overline y+\sqrt{\frac{n_1+n_2}{n_1n_2}}s_wt_{1-\alpha/2}(n_1+n_2-2)]$$

计算得到:

$$[-0.2063,12.2063]$$

(3)

$$[\overline x- \overline y - s_0t_{1-\alpha/2}(l), \overline x- \overline y +s_0t_{1-\alpha/2}(l)]$$

$$s_0^2=\frac{s_x^2}{n_1}+\frac{s_y^2}{n_2}$$

计算可得:

$$[-0.3288,12.3288]$$

(4)

$$[\frac{s_x^2}{s_y^2}*\frac{1}{F_{1-\alpha/2}(n_1-1,n_2-1)},\frac{s_x^2}{s_y^2}*\frac{1}{F_{\alpha/2}(n_1-1,n_2-1)}]$$

计算可得:

$$[0.3359,4.0973]$$



\section{6.6.10}
(1)

$$[\frac{s_x^2}{s_y^2}*\frac{1}{F_{1-\alpha/2}(m-1,n-1)},\frac{s_x^2}{s_y^2}*\frac{1}{F_{\alpha/2}(m-1,n-1)}]$$

计算可得:

$$[0.0620,1.0075]$$

(2)

$$s_w^2=\frac{(m-1)s_x^2+(n-1)s_y^2}{m+n-2}=0.1$$

查表得到$t_{0.975}(18)=2.1009$

计算得到
$$[-0.2771,0.3171]$$

\section{6.6.11}

由指数分布和伽马分布的关系可以得到$\sum^n_{i=1}x_i\sim Ga(n,\lambda)$,根据伽马分布的性质,有:

$$s\lambda\sum^n_{i=1}x_i\sim Ga(n,\frac12)=X^2(2n)$$

因此有:

$$p(X^2_{\frac\alpha2}(2n)\leq2\lambda\sum^n_{i=1}x_i\leq X^2_{1-\frac\alpha2}(2n))=1-\alpha$$

因此置信区间为

$$[\frac{X^2_{\frac\alpha2}(2n)}{2n\overline x},\frac{X^2_{1-\frac\alpha2}(2n)}{2n\overline x}]$$
\end{document}





