\documentclass{article}
\usepackage{ctex}
\usepackage{float}
\usepackage{geometry}
\usepackage{amsmath}
\usepackage{graphicx}
\geometry{a4paper, scale=0.8}

\title{概率论与数理统计第十三次作业}
\author{ZhaohengLi 2017050025}

\begin{document}
\maketitle

\section{5.4.11}
根据题意已知:

$$c(\overline x-\mu_1)\sim N(0,\frac{c^2\sigma^2}{n})$$
$$d(\overline y-\mu_2)\sim N(0,\frac{d^2\sigma^2}{m})$$

$$\frac{(n-1)s_x^2}{\sigma^2}\sim X^2(n-1)$$
$$\frac{(m-1)s_y^2}{\sigma^2}\sim X^2(m-1)$$

因$\overline x,\overline y,s_x^2,s_y^2$的独立性,可以得到:

$$c(\overline x-\mu_1)+d(\overline y-\mu_2)\sim N(0,\frac{c^2\sigma^2}{n}+\frac{d^2\sigma^2}{m})$$

$$\frac{(n+m-2)s_w^2}{\sigma^2}=\frac{(n-1)s_x^2}{\sigma^2}+\frac{(m-1)s_y^2}{\sigma^2}\sim X^2(n+m-2)$$

所以可以得到:

$$t=\frac{c(\overline x-\mu_1)+d(\overline y-\mu_2)}{s_w\sqrt{\frac{c^2}{n}+\frac{d^2}{n}}}=\frac{[c(\overline x-\mu_1)+d(\overline y-\mu_2)]/\sqrt{\frac{c^2\sigma^2}{n}+\frac{d^2\sigma^2}{n}}}{\sqrt{\frac{(n+m-2)s_w^2}{\sigma^2}/(n+m-2)}}\sim t(n+m-2)$$

\section{5.4.12}
根据题意可以知道:

$$x_{n+1}\sim N(\mu,\sigma^2),\overline{x}_n\sim N(\mu,\frac{\sigma^2}{n})$$

$$\frac{(n-1)s_n^2}{\sigma^2}\sim X^2(n-1)$$

$$x_{n+1}-\overline x_n\sim N(0,\sigma^2+\frac{\sigma^2}{n})=N(0,\frac{n+1}n\sigma^2)$$

因此有:

$$t=\frac{(x_{n+1}-\overline x_n)/\sqrt{\frac{n+1}n}}{s_n}=\frac{(x_{n+1}-\overline x_n)/\sqrt{\frac{n+1}n\sigma^2}}{\sqrt{\frac{(n-1)s_n^2}{\sigma^2}/(n-1)}}\sim t(n-1)$$

因此自由度为$n-1$。

\section{5.4.19}

如果$X\sim F(x)$,且$F(x)$为连续严格增函数,那么$Y=F(x)\sim U(0,1)$,$Y=F(x)$的分布函数为:

$$F_Y(y)=P(F(X)\leq y)=P(X\leq F^{-1}(y))=F(F^{-1}(y))=y$$

当$y\leq 0,F_Y(y)=0$,当$y\geq 1,F_Y(y)=1$,所以$F(X)\sim U(0,1)$。

如果$Y\sim U(0,1)$,那么$Z=-lnY\sim X^2(2)$。当$z\leq 0,F_X(z)=0$,当$z>0$,有:

$$F_Z(z)=P(-lnY\leq z)=P(Y\geq e^{-z})=1-e^{-z}$$

这是参数为1的指数分布函数,也是自由度为2的$X^2$
的分布函数,即$Z=-lnY\sim X^2(2)$。

由$X_1,X_2,\cdots,X_n$的相互独立性可以知道$F(X_1),F(X_2),\cdots,F(X_n)$相互独立,因此$u=-2\sum^n_{i=1}F(x_i)\sim X^2(2n)$。


\section{5.5.1}
由几何分布性质可知$T\sim Nb(n,\theta)$,因此可以得到分布列为:

\begin{equation}
P(T=t)=
\begin{pmatrix}
n+t-1\\
t
\end{pmatrix}
\theta^n(1-\theta)^t,\quad t=0,1,2,cdots.
\end{equation}

在给定$T=t$之后,对任意的一个样本$x_1,\cdots,x_n(\sum^n_{i=1}=t)$有:


\begin{equation}
P(X_1=x_1,\cdots,X_n=x_n|T=t)=\frac{P(X_1=x_1,\cdots,X_{n-1}=x_{n-1},X_n=t-\sum^{n-1}_{i=1}x_i)}{P(T=t)}
= \frac{1}{\begin{pmatrix}
n+t-1\\
t
\end{pmatrix}}
\end{equation}

该条件分布与$\theta$无关,因而$T=\sum^n_{i=1}x_i$是充分统计量。

\section{5.5.4}
由题意可以知道条件密度函数为:

$$p_\mu(x_1,\cdots,x_n|T=t)=\frac{p_\mu(x_1,\cdots,x_n)}{p_\mu(t)}=\frac{1}{\sqrt{n}}(2\pi)^{-(n-1)/2}exp\{-\frac12(\sum^n_{i=1}x_i^2-\frac{t^2}{n})\}$$

可以看到它与$\mu$无关,因此$T=\sum^n_{i=1}x_i$是充分统计量。


\section{5.5.5}

样本的联合密度函数为:

$$p(x_1,x_2,\cdots,x_n;\theta)=\theta^n(x_1x_2\cdots x_n)^{\theta-1}$$

令

$$T=\prod^n_{i=1} x_i$$

$$g(t;\theta)=t^{\theta-1}\theta^n, h(x_1,\cdots,x_n)=1$$

由因子分解定理,
$$T=\prod^n_{i=1} x_i$$

为$\theta$的充分统计量。另外$T$的一一变换得到的统计量,如

$$x_1,\cdots,x_n$$

的几何平均

$$(x_1\cdots x_n)^{1/n}$$

或其对数

$$\frac1n\sum^n_{i=1}lnx_i$$

都是$\theta$的充分统计量。

\section{5.5.12}

总体的密度函数为:

\begin{equation}
p(x;\theta)=\left\{
\begin{aligned}
\frac{1}{\theta},&\quad \theta<x<2\theta\\
0,&\quad others
\end{aligned}
\right.
\end{equation}

于是样本的联合密度为:

$$p(x_1,\cdots,x_n;\theta)=(\frac1\theta)^nI_{\theta<x(1)<x(n)<2\theta^n}$$

令$t=x_{(1)},t_2=x_{(n)}$,并取

$$g(t;\theta)=(\frac1\theta)^nI_{\theta<t_1<t_2<2\theta},h(x)=1$$

由因子分解定理,$T(t_1,t_2)=(x_{(1)},x_{(n)})$为$\theta$的充分统计量。

\section{5.5.15}

联合密度函数为:

$$p(x_1,\cdots,x_n;\theta)=(C(\theta))^n exp\{\sum^n_{j=i}\sum^k_{i=1}Q_i(\theta)T_i(x_j)\}\prod^n_{j=1}h(x_j)$$

由分子分解定理可以知道,$T(x)=(\sum^n_{j=1}T_i(x_j),\cdots,\sum^n_{j=1}T_k(x_j))$为充分统计量。

\section{6.1.5}
先求得三个统计量的数学期望:

$$E(\hat \mu_1)=\frac12\mu+\frac13\mu+\frac16\mu=\mu$$

$$E(\hat \mu_2)=\frac13\mu+\frac13\mu+\frac13\mu=\mu$$

$$E(\hat \mu_3)=\frac16\mu+\frac16\mu+\frac23\mu=\mu$$

这说明他们都是总体均值$\mu$的无偏估计,下面求他们的方差,设总体的方差为$\sigma^2$,则:

$$Var(\hat \mu_1)=\frac14\sigma^2+\frac19\sigma^2+\frac1{36}\sigma^2=\frac7{18}\sigma^2$$

$$Var(\hat \mu_2)=\frac19\sigma^2+\frac19\sigma^2+\frac1{9}\sigma^2=\frac13\sigma^2$$

$$Var(\hat \mu_3)=\frac1{36}\sigma^2+\frac1{36}\sigma^2+\frac49\sigma^2=\frac12\sigma^2$$

可以看到$\hat \mu_2$的有效性最好,$\hat \mu_3$最差。

\section{6.1.10}

使用反证法,假设$T(x_1,\cdots,x_n)$为$g(\theta)$的无偏估计,那么:

$$(\frac{1}{\sqrt{2\pi}})^n\int^{\infty}_{-\infty} T(x_1,\cdots,x_n)exp\{-\sum^{n}_{i=1}\frac{(x_i-\theta)^2}{2}\}dx_1\cdots dx_n=|\theta|$$

由上式可以知道,等式左边关于$\theta$处处可导,而等式右边在$\theta=0$的时候不存在导数,因此假设不成立。

因此没有无偏估计。

\section{6.2.4}
(1)

$$E(X)=\frac13\theta$$

因此参数$\theta$的矩估计为$\hat \theta=3\overline x$

(2)

$$E(X)=\frac{\theta+1}{\theta+2}$$

因此参数$\theta$的矩估计为$\hat \theta=\frac{1-2\overline x}{\overline x-1}$

(3)

$$E(X)=\frac{\sqrt \theta}{\sqrt \theta+1}$$


因此参数$\theta$的矩估计为$\hat \theta=(\frac{\overline x}{1-\overline x})^2$

(4)

总体均值和方差计算结果为:

$$E(X)=\theta+\mu,E(X^2)=2\theta^2+2\mu\theta+\mu^2,Var(X)=\theta^2$$

因此参数估计为$\hat\theta=3,\hat\mu=\overline x-s$

\section{6.3.3}

(1)
似然函数:

$$L(\theta)=(\frac{1}{2\theta})^n e^{-\frac{\sum_{i=1}^n|x_i|}{\theta}}$$

对数似然函数为:

$$ln L(\theta)=-nln2\theta-\frac{\sum^n_{i=1}|x_i|}{\theta}$$

对$\theta$求导并令其得零得到似然方程,解得:

$$\hat \theta = \frac{\sum^n_{i=1}|x_i|}{n}$$

因为

$$\frac{\partial lnL(\theta)}{\partial \theta^2}|_{\hat\theta}=\frac{n^3}{(\sum|x_i|)^2}<0$$

因此$\hat \theta $是$\theta$的最大似然估计。

(2)

此处的似然函数为:

$$L(\theta)=I_{\{\theta-\frac12<x_{(1)}<x_{(n)}<\theta+\frac12\}}$$

他只有两个取值,$0,1$,为了使似然函数取1,$\theta$的取值范围是$x_{(n)}-\frac12<\theta<x_{(1)}+\frac12$,因此$\theta$的最大似然估计$\hat \theta$可以取范围中的任意值,这说明MLE可能不止一个。

(3)

似然函数为:

$$L(\theta)=\frac1{(\theta_2-\theta_1)^n}I_{\{\theta_1<x_{(1)}<x_{(n)}<\theta_2}\}$$

要使$L(\theta)$尽可能大,那么先示性函数应为1,这说明$\theta_1<x_{(1)}<x_{(n)}<\theta_2$;其次$\theta_2-\theta_1$要尽量小,综上可知,$\theta_1$的最大的然估计应为$x_{(1)}$,$\theta_2$的最大似然估计为$x_{(n)}$。




\section{6.3.7}
(1)
根据题意可以求得
$$E(\overline x)=\frac{3\theta}{2},Var(\overline x)=\frac{\theta^2}{12n}
$$

因为$E(\hat \theta)=\frac23E(\overline x)=\theta$,这说明$\hat \theta = \frac23\overline x$是参数$\theta$的无偏估计,因此可以得到:

$$Var(\hat \theta)=\frac{\theta^2}{27n}\rightarrow 0$$

因此相合估计得证。

(2)
根据题意,最大似然估计为:

$$\hat \theta_{mle}=\frac{x_{(n)}}2$$

进一步求得均值和方差分别为:

$$E(x_{(n)})=\frac{2n+1}{n+1}\theta$$
$$Var(x_{(n)})=\frac{n\theta^2}{(n+2)(n+1)^2}$$

可以看到:

$$E(\hat \theta)=\frac{2n+1}{2(n+1)}\theta\rightarrow\theta(n\rightarrow+\infty)$$

$$Var(\hat\theta)=\frac{n\theta^2}{4(n+1)^2(n+2)}\rightarrow0(n\rightarrow+\infty)$$

因此不是无偏估计,是相合估计。

\section{6.3.9}

根据题意求得均方误差为:

$$MSE(\hat\theta_a)=Var(a\overline x)+(E(a\overline x)-\theta)^2=\frac{a^2\theta^2}{n}+(a-1)^2\theta^2$$

对a求导并令其为0,可以得到当$a_0=\frac{n}{n+1}$时,上式取得最小值,并且有:

$$MSE(\hat\theta_{a_0})=\frac1{n+1}\theta^2<\frac1n\theta^2=MSE(\overline x)$$

这证明了在均方误差准则下存在一个优于$\overline x$的估计,这也说明,有偏估计并不是总比无偏估计差。

\end{document}





