\documentclass{article}
\usepackage{ctex}
\usepackage{float}
\usepackage{geometry}
\usepackage{amsmath}
\usepackage{graphicx}
\geometry{a4paper, scale=0.8}

\title{概率论与数理统计第四次作业}
\author{ZhaohengLi 2017050025}

\begin{document}
\maketitle

\section{2.2.8}
当回答顺序为1,2时,X分布列为:
\begin{table}[H]
\centering
\begin{tabular}{l|lll}
\hline
X & 0   & 200     & 300     \\ \hline
P & 0.4 & 0.6*0.2 & 0.6*0.8 \\ \hline
\end{tabular}
\end{table}

$E(X)=168$

当回答顺序为2,1时,X分布列为:
\begin{table}[H]
\centering
\begin{tabular}{l|lll}
\hline
X & 0   & 100     & 300     \\ \hline
P & 0.2 & 0.8*0.4 & 0.8*0.6 \\ \hline
\end{tabular}
\end{table}

$E(X)=176$

因此应该先回答问题2。
\section{2.2.10}
设X为保险公司的收益,则X的分布列为:
\begin{table}[H]
\centering
\begin{tabular}{l|ll}
\hline
X & -a & ka  \\ \hline
P & p  & 1-p \\ \hline
\end{tabular}
\end{table}

所以保险公司的期望收益为$E(X)=-ap+ka(1-p)$,由$E(X)\leq 0.1a$,可得到$-ap+ka(1-p)\leq 0.1a$,解得:
$$k\leq \frac{0.1+p}{1-p}$$

一些具体的数值可查下表:
\begin{table}[H]
\centering
\begin{tabular}{llllllll}
X & 0.01   & 0.05    & 0.1    & 0.2    & 0.3    & 0.4   & 0.5    \\
P & 0.1111 & 0.11579 & 0.2222 & 0.3750 & 0.5714 & .8333 & 1.2000
\end{tabular}
\end{table}

\section{2.2.14}

由分布函数可得到密度函数为:
\begin{equation}
p(x)=
\left\{
\begin{aligned}
&0.5e^x,&x<0,\\
&0,&0\leq x <1,\\
&0.25e^{-0.5(x-1)},&x\leq1
\end{aligned}
\right.
\end{equation}


$$E(X)=\int_{-\infty}^00.5xe^xdx+\int_1^{+\infty}0.25xe^{-0.5(x-1)}dx=1$$
\section{2.2.17}
$$p=P(X>\pi/3)=\int_{\pi/3}^\pi0.5cos(0.5x)dx=1-sin(\pi/6)=0.5$$
$$P(Y=k)=C^k_4*0.5^k0.5^{4-k},\quad k=0,1,2,3,4.$$
$$E(Y^2)=\sum^4_{k=0}k^2C^k_40.5^4=80*0.5^4=5$$
\section{2.2.20}
$$E(X)=\int^{+\infty}_{-\infty}xp(x)dx=\int_{-\infty}^0xp(x)dx+\int^{+\infty}_0xp(x)dx$$
\begin{equation}
\begin{aligned}
\int_{-\infty}^0xp(x)dx&=-\int^0_{-\infty}(\int^0_xdy)p(x)dx\\
&=-\int^0_{-\infty}\int^y_{-\infty}p(x)dxdy\\
&=-\int^0_{-\infty}F(y)dy
\end{aligned}
\end{equation}

\begin{equation}
\begin{aligned}
\int^{+\infty}_0xp(x)dx&=\int_0^{+\infty}(\int_0^xdy)p(x)dx\\
&=\int_0^{+\infty}(\int_y^{+\infty}p(x)dx)dy\\
&=\int_0^{+\infty}[1-F(y)]dy
\end{aligned}
\end{equation}

两式相加即可得到要证明的等式。

\section{2.3.7}

由题意可得:

$$1=\int_{-\infty}^{+\infty}p(x)dx=a/2+b/3$$
$$0.5=E(X)=\int_0^1x(ax+bx^2)dx=a/3+b/4$$

解上述两式可得到$a=6,b=-6$。
$$E(X^2)=\int^1_0x^2(6x-6x^2)dx=0.3$$
$$Var(X)=E(X^2)-[E(X)]^2=0.05$$

\section{2.3.11}



$$Var(X)\leq E(X-\frac{x_1+x_n}{2})^2\leq E(x_n-\frac{x_1+x_n}{2})^2=(\frac{x_n-x_1}{2})^2$$


\section{2.3.12}

\begin{equation}
\begin{aligned}
P(X>\epsilon)&=\int_\epsilon^{+\infty}p(x)dx\\
&\leq\int_\epsilon^{+\infty}\frac{g(x)}{g(\epsilon)}p(x)dx\\
&\leq\int_{-\infty}^{+\infty}\frac{g(x)}{g(\epsilon)}p(x)dx\\
&=\frac{E(g(X))}{g(\epsilon)}
\end{aligned}
\end{equation}

\section{2.4.3}
$$P(X\geq2)=P(X=2)+P(X=3)=3*0.7^2*0.3+0.7^3=0.784$$

\section{2.4.10}
设事件A为“服用此药后,一年感冒两次”,事件B为“服用此药后有效”。根据题意有:

$$P(A)=P(B)P(A|B)+P(\overline B)P(A|\overline B)$$
$$P(B|A)=\frac{0.75*\frac{3^2}{2!}e^{-3}}{0.75*\frac{3^2}{2!}e^{-3}+0.25*\frac{5^2}{2!}e^{-5}}=0.889$$
\end{document}