\documentclass{article}
\usepackage{ctex}
\usepackage{geometry}
\usepackage{amsmath}
\geometry{a4paper, scale=0.8}

\title{概率论与数理统计第五次作业}
\author{ZhaohengLi 2017050025}

\begin{document}
\maketitle

\section{2.5.6}
设$X$为圆盘的直径,则圆盘的面积为$Y=\pi \frac{X^2}{4}$,所以平均面积计算如下。

$$E(X^2)=\int_a^bx^2p(x)dx=\int_a^bx^2\frac{1}{b-a}dx=\frac{1}{3}(a^2+b^2+ab)$$
$$E(Y)=\frac{\pi}{4}E(X^2)=\frac{\pi}{12}(a^2+b^2+ab)$$

\section{2.5.10}
设设备在一年内损坏的概率为$p$,则:

$$p=P(X\leq 1)=\int_0^10.25e^{-0.25x}dx=0.2212$$

设每台设备的利润为$Y$,那么:

$$E(Y)=100-300p=33.64$$

因此每台设备的平均利润为33.64元。

\section{2.5.11}
因为每次来到银行发生的事情是独立的,因此$Y\sim B(5,p)$,其中$p=P(X>10)=e^{-2}$。因此有:

$$P(Y\geq 1)=1-P(Y<1)=1-P(Y=0)=1-(1-p)^5=1-(1-e^{-2})^5=0.5167$$

\section{2.5.17}
(1)由题目可以得到:

$$P(X\leq 70)=0.5=\Phi (\frac{70-\mu}{\sigma})$$
$$P(X\leq 60)=0.25=\Phi (\frac{60-\mu}{\sigma})$$

查表解得$\mu = 70,\sigma = 14.81$


(2)设$Y$为5名男子中体重超过65kg的男子数量,则$Y\sim B(5,p)$,其中;

$$p=P(X\geq65)=1-P(X<65)=1-\Phi(\frac{65-70}{14.81})=\Phi(0.3376)=0.6324$$

则有:

$$P(Y\geq2)=1-P(X<2)=1-(1-p)^5-5*p*(1-p)^4=0.94$$

\section{2.5.19}

由题意可得:

$$P(X>96)=0.023=1-\Phi(\frac{96-\mu}{\sigma})=1-\Phi(\frac{96-72}{\sigma})$$

通过反查表可得到:

$$\frac{96-72}{\sigma}=2$$

解得$\sigma=12$,从而得到$X\sim N(72,12^2)$。

$$P(60<X<84)=\Phi(\frac{84-72}{12})-\Phi(\frac{60-72}{12})=2\Phi(1)-1=0.6826$$

\section{2.5.30}

$$E|X-\mu|=\frac{1}{\sqrt{2\pi}\sigma}\int_{-\infty}^{+\infty}|x-\mu|e^{-\frac{(x-\mu)^2}{2\sigma^2}}dx$$

利用$t=\frac{x-\mu}{\sigma}$,化简为:

$$E|X-\mu|=\sigma\sqrt{\frac{2}{\pi}}\int_0^{+\infty}e^{-\frac{t^2}{2}}d(\frac{t^2}{2})=\sigma\sqrt{\frac{2}{\pi}}$$

\section{2.5.32}

$$P(X<4)=0.25*\int^{4}_{0}\frac{0.25}{\Gamma(2)}xe^{-\frac{x}{2}}dx=0.25*\int^{4}_{0}0.25*xe^{-\frac{x}{2}}dx=0.5940$$
\section{补充}
几何分布的分布列函数为:

$$P(X=k)=(1-p)^{k-1}p,\quad k=1,2,\cdots.$$

其矩母函数函数为:

$$M(t)=\sum_{k=1}^{\infty}(1-p)^{k-1}pe^{tk}=\frac{pe^t}{1-(1-p)e^t}$$

三阶矩为:

$$E(X^3)=M'''(0)={{1}\over{p}}-{{6}\over{p^2}}+{{6}\over{p^3}}$$

标准正态分布的密度函数为:

$$\phi (u)=\frac1{\sqrt{2\pi}}e^{-\frac{u^2}{2}},\quad -\infty<u<+\infty$$

其矩母函数函数为:

$$M(t)=\int_{-\infty}^{+\infty}\frac1{\sqrt{2\pi}}e^{-\frac{u^2}{2}}e^{tu}du=e^{\frac{t^2}{2}}$$

三阶矩为:
$$M'''(t) = \left(t^3+3\,t\right)\,e^{{{t^2}\over{2}}}$$

$$E(X^3)=M'''(0)=0$$

\end{document}


