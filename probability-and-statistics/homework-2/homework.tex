\documentclass{article}
\usepackage{ctex}
\usepackage{geometry}
\usepackage{amsmath}
\geometry{a4paper, scale=0.8}

\title{概率论与数理统计第二次作业}
\author{ZhaohengLi 2017050025}

\begin{document}
\maketitle
\section{1.4.9}
由条件概率的定义可得
$$P(B|A\cup \overline{B})=\frac{P(AB)}{P(A\cup \overline{B})}$$
已知
$$P(A \cup \overline B)=P(A)+P(\overline B)-P(A\overline B)=0.7+0.6-0.5=0.8$$
$$P(AB)=P(A)-P(A\overline B)=0.7-0.5=0.2$$
因此
$$P(B|A\cup \overline{B})=\frac{P(AB)}{P(A\cup \overline{B})}=\frac{0.2}{0.8}=0.25$$

\section{1.4.13}
(1)本题考察全概率公式的使用,设事件A为“从乙口袋中取出的球为白球”。
$$P(A)=\frac{a}{a+b}\cdot\frac{n+1}{n+m+1}+\frac{b}{a+b}\cdot\frac{n}{n+m+1}$$


(2)
\begin{equation}
\begin{aligned}
P(A)=&\frac{a(a-1)}{(a+b)(a+b-1)}\cdot\frac{n+2}{n+m+2}\\
+&\frac{2ab}{(a+b)(a+b-1)}\cdot\frac{n+1}{n+m+2}\\
+&\frac{b(b-1)}{(a+b)(a+b-1)}\cdot\frac{n}{n+m+2}
\end{aligned}
\end{equation}

\section{1.4.17}
(1)本小题考察全概率公式的使用,设事件$A_{i}$为“第$i$次取出的是一等品”,$i=1,2.$。设事件$B_{i}$为“从第$i$箱中取球“。则有
$$P(A_{1})=P(B_{1})P(A_{1}|B_{1})+P(B_{2})P(A_{1}|B_{2})=0.5$$


(2)在前一小题的基础上,本小题加入了条件概率的公式,有
$$P(A_{2}|A_{1})=\frac{P(A_{1}A_{2})}{P(A_{1})}$$
$$P(A_{1}A_{2})=P(B_{1})P(A_{1}A_{2}|B_{1})+P(B_{2})P(A_{1}A_{2}|B_{2})=0.2534$$
$$P(A_{2}|A_{1})=\frac{P(A_{1}A_{2})}{P(A_{1})}=\frac{0.2534}{0.5}=0.5068$$

\section{1.4.20}
本题考察贝叶斯公式,设事件$A$为“取出的是白球”,事件$B$为“原来的是白球”,则有
$$P(B|A)=\frac{P(B)P(A|B)}{P(B)P(A|B)+P(\overline B)P(A|\overline B)}=\frac{0.5*1}{0.5*1+0.5*0.5}=\frac{2}{3}$$


\section{1.4.23}
本题考察全概率公式的使用。设事件$A_{i}$为“第$i$次由甲掷骰子“,依题意可得到
$$P(A_{i})=P(A_{i-1})\frac{5}{6}+P(\overline{A_{i-1}})\frac{1}{6}$$
$$P(A_{1})=1$$
由此可得到递推公式
$$P(A_{i})-\frac{1}{2}=\frac{2}{3}(P(A_{i-1})-\frac{1}{2}) \quad n\ge2$$
整理可得到
$$P(A_{n})=\frac{1}{2}[1+(\frac{2}{3})^{n-1}] \quad n=2,3,\cdots.$$

\section{1.5.3}
设事件$A$为“目标被击中”,事件$B_{1}$为“甲射中目标”,事件$B_{2}$为“乙射中目标,则有
$$P(A)=P(B_{1}\cup B_{2})=P(B_{1})+P(B_{2})-P(B_{1}B_{2})=0.94$$
$$P(B_{1}|A)=\frac{P(AB_{1})}{P(A)}=\frac{P(B_{1})}{P(A)}=\frac{0.8}{0.94}=0.851$$

\section{1.5.8}
$$P(A\cup B)=P(A)+P(B)-P(AB)$$

(1)当$AB$不相容时,$P(AB)=0$,因此$P(B)=0.5$。

(2)当$AB$独立时,$P(AB)=P(A)*P(B)$,因此$P(B)=\frac{5}{6}$。

(3)当$A\subset B$时,$P(B)=P(A\cup B)=0.9$。

\section{1.5.10}
由题意可知
$$P(\overline A B)=P(A\overline B)=\frac{1}{4}$$

又因为$AB$独立
$$P(A)-P(A)P(B)=\frac{1}{4}$$
$$P(B)-P(A)P(B)=\frac{1}{4}$$

得到$P(A)=P(B)=0.5$。


\section{1.5.16}

射事件$A,B,C$分别为“猎人在$100m,150m,200m$处击中猎物”,设事件$D$为“最终击中猎物”,因为击中概率与距离成反比,则由$P(A)=1/2$可知$P(B)=1/3$以及$P(C)=1/4$,根据题意可得
$$P(D)=P(A)+P(\overline A B)+P(\overline A \overline B C)=\frac{3}{4}$$

\end{document}