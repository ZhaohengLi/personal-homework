\documentclass{article}

\usepackage{ctex}
\usepackage{listings}
\usepackage[framed,numbered,autolinebreaks,useliterate]{mcode}
\usepackage{geometry}
\usepackage{multirow}
\usepackage{graphicx}
\usepackage{amsmath}
\usepackage{float}
\geometry{a4paper, scale=0.8}

\title{数学实验实验报告}
\author{ZhaohengLi 2017050025\\cainetatum@foxmail.com\\15801206130}

\begin{document}
\maketitle
\section{实验目的}
\begin{itemize}
	\item{掌握使用 LINGO 解整数规划模型的方法;}
	\item{通过求解实际问题,学会建立实际问题的整数规划模型。}
\end{itemize}


\section{CH10-T8 服务员聘用}

\subsection{模型建立}
设储蓄所中以中午12时到下午1时为午餐时间的全时服务员有 $x_1$名,以下午1时到下午2时为午餐时间的全时服务员有$x_2$名。因为半时服务员必须在储蓄所连续工作4小时,因此设上午9时开始工作的半时服务员有$y_1$名,上午10时开始工作的半时服务员有$y_2$名,上午11时开始工作的半时服务员有$y_3$名,中午12时开始工作的半时服务员有$y_4$名,下午1时开始工作的半时服务员有$y_5$名,显然,$x_i,y_j\in Z^+,\quad (i=1,2.\quad j=1,2,\cdots,5.)$。

设$x,y$分别为储蓄所雇佣的全时服务员和半时服务员的数量,因此有:

$$x=x_1+x_2$$
$$y=y_1+y_2+y_3+y_4+y_5$$

依题意,储蓄所各个时段的服务员数量不能够低于表格中要求的数量,因此有:

$$x+y_1 \geq 4$$
$$x+y_1+y_2 \geq 3$$
$$x+y_1+y_2+y_3 \geq 4$$
$$x+y_1+y_2+y_3+y_4-x_1 \geq 6$$
$$x+y_2+y_3+y_4+y_5-x_2 \geq 5$$
$$x+y_3+y_4+y_5 \geq 6$$
$$x+y_4+y_5 \geq 8$$
$$x+y_5 \geq 8$$

如果要求每天雇佣的半时服务员不超过3个,则有:

$$y\leq3$$

如果要求不可以雇佣半时服务员,则有:

$$y\leq0$$

如果对半时服务员的雇佣数量没有限制,则不对$y$进行限制。

为了最小化报酬总数,优化目标设为:

$$min z = 100x+40y$$

\subsection{算法设计}

该问题需要使用整数规划模型来求解这个问题,因此可以使用 LINGO 的 @gin关键字为相关的变量添加整数优化,从而求解本问题。

\begin{lstlisting}
model:

min = 100 * x + 40 * y;

x = x1 + x2;
y = y1 + y2 + y3 + y4 + y5;

x + y1 > 4;
x + y1 + y2 > 3;
x + y1 + y2 + y3 > 4;
x + y1 + y2 + y3 + y4 - x1 > 6;
x + y2 + y3 + y4 + y5 - x2 > 5;
x + y3 + y4 + y5 > 6;
x + y4 + y5 > 8;
x + y5 >8;

y < 3;

@gin(x);@gin(x1);@gin(x2);@gin(y);@gin(y1);@gin(y2);@gin(y3);@gin(y4);@gin(y5);

end
\end{lstlisting}

\subsection{结果分析}

当每天雇佣的半时服务员不超过3个时,计算得出:

$$x_1=3,x_2=4,y_1=0,y_2=0,y_3=2,y_4=0,y_5=1$$

最小的费用为820元。

当不可以雇佣半时服务员的时候,计算得出:

$$x_1=5,x_2=6,y_1=0,y_2=0,y_3=0,y_4=0,y_5=0$$

最小费用为1100元,比原来每天增加了280元。

当雇佣半时服务员的数量没有限制时,计算得出:

$$x_1=0,x_2=0,y_1=4,y_2=0,y_3=0,y_4=2,y_5=8$$

最小费用为560元,比原来每天减少了260元。



\section{CH10-T9 原油采购与加工}
\subsection{模型建立}

假设原油 A 使用存货中的 at 进行加工甲、乙两种汽油,又购进新原油 A 总共 tt,而最 终在生产时,甲汽油含有原油 A 总共 et,乙汽油含有原油 A 总共 f t;原油 B 中使用 ct 加 工甲汽油,使用 dt 加工乙汽油。根据题目中的约束,可以列出下列条件:

$$a\leq 500,\quad t\geq 0 $$
$$c+d\leq1000$$
$$e+f=a+t$$
$$e\geq0.5(e+c),\quad f\geq0.6(f+d)$$

需要注意的是,原油 A 的使用上,并没有添加“必须先使用存货”的限制,这是因为最 终的最优结果一定会倾向于使用存货,因为不会花费多于的财产,对结果没有影响。

考虑到新购进的原油价格随着购买吨数的不同有着不同的价格,其曲线为分段函数的形 状,因此考虑将其分成三段 t1, t2, t3,分别代表购进吨数 500 吨以内的部分,超过 500 吨但 不超过 1000 吨的部分,以及超过 1000 吨的部分。考虑到实际购买情况,列出该分段函数相 关的下列约束关系:

$$t=t_1+t_2+t_3$$
$$t_1\leq500,\quad t_2\leq500,\quad t_3\leq500$$
$$t_2(t_1-500)=0$$
$$t_3(t_2-500)=0$$


最终的盈利最大即为优化目标,如下所示:


$$max z = 4800(e+c)+5600(f+d)-10000t_1-8000t_2-6000t_3$$

\subsection{算法设计}
题目要求使用连续规划和整数规划两种模型来求解这个问题,因此可以使用 LINGO 的 @gin关键字为相关的变量添加整数优化,从而求解本问题。

连续规划LINGO代码:

\begin{lstlisting}
model:

max = 4800 * (e + c) + 5600 * (f + d) − 10000 * t1 − 8000 * t2 − 6000 * t3;

a < 500;
e > 0.5 * (e + c);
f > 0.6 * (f + d);
c + d < 1000;
e + f = a + t;

t1 < 500;
t2 < 500;
t3 < 500;
t = t1 + t2 + t3;

t2 * (t1 − 500) = 0;
t3 * (t2 − 500) = 0;

end
\end{lstlisting}

整数规划LINGO代码:

\begin{lstlisting}
model:
max = 4800 * (e + c) + 5600 * (f + d) − 10000 * t1 − 8000 * t2 − 6000 * t3;

a < 500;
e > 0.5 * (e + c);
f > 0.6 * (f + d);
c + d < 1000;
e + f = a + t;

t1 < 500;
t2 < 500;
t3 < 500;
t = t1 + t2 + t3;

t2 * (t1 − 500) = 0;
t3 * (t2 − 500) = 0;

@gin(a); @gin(e); @gin(c); @gin(f); @gin(d);
@gin(t1); @gin(t2); @gin(t3);

end
\end{lstlisting}

\subsection{结果分析}

使用连续规划求解的结果如下:

$$a=500.0,e=0.0,c=0.0,f=1500.0,d=1000.0,t=1000.0$$

最优值为$5000000.0$。

使用整数规划求解的结果如下:

$$a=500,e=0,c=0,f=1500,d=1000,t=1000$$

最优值为$5000000$。

对比连续规划和整数规划能够看出,二者求出的结果完全相同,对应的最优值也相同, 但是就求解过程来看,整数规划使用的是 MINLP 模型,迭代轮数为 938;而连续规划使用 的是 NLP 模型,迭代轮数为 206,远远少于整数规划,但最终结果也基本为整数。可以看 出,整数规划在求解方面的确效率不如普通的连续规划。


另外,实际的最终结果表明不需要生产甲汽油。这说明了单单对比甲乙两种汽油的产出 投入比,显然乙汽油能够花费少的原料产生更多的利润。尽管生产单位的甲汽油所需要的原 油 A 较少,但是其卖出的售价也相应的低很多,利润赶不上乙汽油。


\subsection{结论}

该公司应该购买原油 A 总共 1000t,并且使用所有的原油 A 和原油 B 全部用来 生产乙汽油,能够使得利润最大,为 5000000 元。


\section{CH10-T11 钢管下料}

\subsection{模型建立}
假设使用了这四种切割模式,他们切割的钢管数目分别为 $x_1, x_2, x_3, x_4$,且保证 $x_1 \leq x_2 \leq x_3 \leq x_4$。并设第 i 种切割模式下每根原料钢管生产长 290mm, 315mm, 350mm, 455mm 钢管各 $r_{1i}, r_{2i}, r_{3i}, r_{4i}$ 根。

为了生产满足用户需求数目的钢管,需要的约束条件为:

$$x_1r_{11}+x_2r_{12}+x_3r_{13}+x_4r_{14}\geq 15$$
$$x_1r_{21}+x_2r_{22}+x_3r_{23}+x_4r_{24}\geq 28$$
$$x_1r_{31}+x_2r_{32}+x_3r_{33}+x_4r_{34}\geq 21$$
$$x_1r_{41}+x_2r_{42}+x_3r_{43}+x_4r_{44}\geq 30$$
$$\sum_ir_{ij}\leq5,\quad \forall j$$


题目中给出的余料限制是每种切割模式下的余料浪费不能超过100mm,因此增加约束条件为:

$$1750\leq 290r_{11}+315r_{21}+350r_{31}+455r_{41}\leq1850$$
$$1750\leq 290r_{12}+315r_{22}+350r_{32}+455r_{42}\leq1850$$
$$1750\leq 290r_{13}+315r_{23}+350r_{33}+455r_{43}\leq1850$$
$$1750\leq 290r_{14}+315r_{24}+350r_{34}+455r_{44}\leq1850$$

为了使总费用最少,即是使得购买钢管所花费的钱数和加工所需要的钱数总和最小,假
设一根钢管的价值为 1,即优化目标为:

$$minz=1.1x_2+1.2x_2+1.3x_3+1.4x_4$$

\subsection{算法设计}

\begin{lstlisting}
model:

sets:
nn/1..4/: req, w, x, len;
  link(nn, nn): r;
endsets

data:
w = 1.1 1.2 1.3 1.4;
req = 15 28 21 30;
len = 290 315 350 455;
enddata

min = @sum(nn: w * x);

@for(nn(i):
  @sum(nn(j): x(j) * r(i, j)) >= req(i);
  @sum(nn(j): r(j, i)) <= 5;
  @sum(nn(j): len(j) * r(j, i)) >= 1750;
  @sum(nn(j): len(j) * r(j, i)) <= 1850;
  @gin(x(i));
  @for(nn(j): @gin(r(i, j)));
);

end
\end{lstlisting}

\subsection{结果分析}
通过LINGO模型计算,可以得到切割方式如下:
\begin{table}[H]
\centering
\begin{tabular}{|l|l|l|l|l|}
\hline
切割方式 & 290mm & 315mm & 350mm & 455mm \\ \hline
1    & 1     & 2     & 0     & 2     \\ \hline
2    & 0     & 0     & 5     & 0     \\ \hline
3    & 2     & 0     & 1     & 2     \\ \hline
4    & 0     & 3     & 1     & 1     \\ \hline
\end{tabular}
\end{table}
每种方式切割钢管数量为:

$$x_1=14,x_2=4,x_3=1,x_4=0$$

最后得出的最小费用为21.5。

由求解结果可以看出,第四种切割方式实际上并没有被使用,所以得出的第四种切割方 式实际上也没有意义,在给出最终结果的时候应该舍去。

\subsection{结论}

为了使总费用最小,应该采用表 2所示的三种切割方式。使用第一种切割
方式购进并切割 14 根钢管、使用第二种方式购进并切割 4 根钢管、使用第三种方式购进并 切割 1 根钢管,即可以在满足用户需求的前提下花费最少的费用,所得到的最小费用为 21.5 倍个钢管价值。


\begin{table}[H]
\centering
\begin{tabular}{|l|l|l|l|l|}
\hline
切割方式 & 290mm & 315mm & 350mm & 455mm \\ \hline
1    & 1     & 2     & 0     & 2     \\ \hline
2    & 0     & 0     & 5     & 0     \\ \hline
3    & 2     & 0     & 1     & 2     \\ \hline

\end{tabular}
\end{table}

\section{实验总结}

通过这次的实验,我学会了 LINGO 软件的基本使用和编程技巧,并对整数规划有了更 深的理解。希望在之后的课堂上老师能够当堂进行相关的技巧演示并给出题目的分步解答。



\end{document}